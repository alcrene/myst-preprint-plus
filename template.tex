\documentclass[[#- if options.two_column -#]twocolumn,switch,[#- endif -#]
               [#- if options.papersize == "letter" -#]letterpaper
               [#- elif options.papersize == "a4" -#]a4paper
               [#- endif -#]
               ]{article}
%# The 'switch' option is picked up by certain packages, notably lineno, #%
%# to determine whether to switch the side of the output on every page   #%
%# Normally it should be true for two-column and double-sided output.    #%
\PassOptionsToPackage{style}{abstract}
\usepackage[[#- if options.status -#]       status={[-options.status-]},
            [# endif -#]
            [#- if options.venue_status -#] venue={[-options.venue_status-]},
            [# endif -#]
            runningtitle={[- doc.short_title or doc.title -]},
            geom=[- options.text_geometry -],
            [#- if options.margin_ticks -#] marginticks,
            [# endif -#]
            noabstracttitle,
            floatplacement=medium,
            suppfloatplacement=extratight,  % Make it easier to place floats alongside text in Supplementary, which tends to have a higher float/text ratio
            supplementaryprefix=[-options.supplementary_prefix-],
            orcid=true,        % Define the \orcid macro
            ]{preprint+}

\usepackage[utf8]{inputenc} % allow utf-8 input
\usepackage[T1]{fontenc}
[# if options.language #]
\usepackage[[- options.language -]]{babel}
[# endif #]

[-IMPORTS-]

\usepackage{xcolor}
\usepackage[switch]{lineno}         % Line numbers; 'switch' for two column layout
\usepackage{datetime2}
\usepackage{datetime2-calc}

% (abstract package is loaded by preprint+)
\renewcommand{\abstractnamefont}{\normalfont\normalsize\bfseries}
\renewcommand{\abstracttextfont}{\normalfont\normalsize}
\renewcommand{\abstitlestyle}[1]{}  % Don’t print "Abstract" title


\usepackage{doi}  % Automatically converts DOIs in references into links
\renewcommand{\doitext}{https://doi.org/}  % DataCite recommends full URLs: https://datacite.org/blog/cool-dois/

%% Bibliography options


%# We allow for using either BibTex or BibLaTeX                             #%
%# Most journals provide BibTeX style files (.bst), but BibLaTeX can        #%
%# often produces better output for less effort.                            #%
%# Spending time hand-adjusting the output with BibTex is obviously         #%
%# worthwhile if the article is being resubmitted after positive reviews;   #%
%# Otherwise I would use BibLaTeX for publishing on arXiv or a webpage.     #%
[# if options.use_biblatex #]
%# USE BIBLATEX  #%
% NB: We include DOIs even though Nature normally only prints them for software, for two reasons:
% 1. Otherwise we need to find another way to force the printing of at least software DOIs.
% 2. For our self-published versions we want to include it (clicking a link is easy)
% NB: Currently arXiv references are not exported with biblatex fields
% eprint, eprinttype, so the eprint and arxiv options don’t do anything.
\usepackage[citestyle=[-options.cite_style-],
            bibstyle=[-options.bib_style or options.cite_style-],
            autocite=[-options.citep_format-],
            pluralothers=true,
            natbib=true,
            url=false,doi=true,eprint=true,arxiv=abs
            ]{biblatex}
\DeclareFieldFormat{doi}{doi\addcolon\space\url{#1}}  % Add a space after 'doi:'
%# \setcounter{biburllcpenalty}{7000}   % Allow biblatex to break URLs #%
%# \setcounter{biburlucpenalty}{8000}   % Prefer breaks after lowercase letters #%
%# \setcounter{biburlnumpenalty}{9000}  % c.f. https://tex.stackexchange.com/a/134281  #%
[# for item in doc.bibliography #]
\addbibresource{[- item -]}
[# endfor #]
[# endif #]

% Section title spacing  options
\usepackage{titlesec}
\titlespacing\section{0pt}{12pt plus 3pt minus 3pt}{1pt plus 1pt minus 1pt}
\titlespacing\subsection{0pt}{10pt plus 3pt minus 3pt}{1pt plus 1pt minus 1pt}
\titlespacing\subsubsection{0pt}{8pt plus 3pt minus 3pt}{1pt plus 1pt minus 1pt}
\titleformat{\part}[display]{\filcenter\LARGE\bfseries}{}{0pt}{}{}  % Used to introduce supplementary material
\titleformat*{\section}{\large\bfseries\raggedright}
\titleformat*{\subsection}{\normalsize\bfseries\raggedright}
\titleformat*{\subsubsection}{\normalsize\itshape\raggedright}


% hyperref should generally be loaded as late as possible
\usepackage{hyperref}
% colors for hyperlinks
\hypersetup{pdftitle={[-doc.title-]},
            pdfauthor={[--doc.authors|join(', ', 'name')--]},
            colorlinks=true, linkcolor=purple, urlcolor=blue, citecolor=cyan, anchorcolor=black}

%% Configure cleveref
%% (Generally a good idea to import cleveref last)
\usepackage{cleveref}
%% Use a more compact format for compressing equations
% \newcommand{\crefrangeconjunction}{--}
%% Use the following to spell out "equation" in cross-references
%\crefname{equation}{equation}{equations}
%\crefname{pluralequation}{equations}{equations}
%\creflabelformat{pluralequation}{(#2#1#3)}
[# if parts.supplementary #]
%% Create new crossref types for Supplementary, with "Supp" in the name
%% The types are activated after \supplementary below, with \crefalias
\crefname{sifigure}{[- supplementary_prefix -] Fig.}{[- supplementary_prefix -] Figs.}
\crefname{sitable}{[- supplementary_prefix -] Table}{[- supplementary_prefix -] Tables}
[# endif #]


%%%%%%%%%   Typesetting params   %%%%%%%%%%
% See https://www.tug.org/TUGboat/tb14-2/tb39taylor-para.pdf
%     https://tex.stackexchange.com/a/50850

\emergencystretch=2pt
% \pretolerance = 150
% \tolerance = 9999
% \hbadness = 150
\hfuzz=4pt
\vfuzz=1pt

[# if options.two_column #]
% Reduce the amount of indentation for two column layouts
% Lists use \leftmargini, \leftmarginii, etc.
% Quotes uses \leftmargin, which by default is set to \leftmargini
\setlength\leftmargin{1.5em}
\setlength\leftmargini{2em}
[# endif #]

\usepackage{xurl}   % Load xurl after biblatex to automatically fix url line breaks in the bibliography
                    % Load xurl before hyperref (https://tex.stackexchange.com/a/479390)
                    
%%%%%%%%%%%%%%%%%%%%%%%%%%%%%%%%%%%%%%%%%%%


%%%%%%%%%%%%%%%%   Watermark   %%%%%%%%%%%%%%%%

% Add watermark with submission status
% Awaiting watermark support
% \usepackage{xwatermark}
% % Left watermark
% \newwatermark[firstpage,color=gray!60,angle=90,scale=0.32, xpos=-4.05in,ypos=0]{\href{https://doi.org/}{\color{gray}{Publication doi}}}
% % Right watermark
% \newwatermark[firstpage,color=gray!60,angle=90,scale=0.32, xpos=3.9in,ypos=0]{\href{https://doi.org/}{\color{gray}{Preprint doi}}}
% % Bottom watermark
% \newwatermark[firstpage,color=gray!90,angle=0,scale=0.28, xpos=0in,ypos=-5in]{*correspondence: \texttt{}}

%%%%%%%%%%%%%%    Front matter    %%%%%%%%%%%%%%

\usepackage{authblk}
\renewcommand*{\Authfont}{\bfseries}
\title{[-doc.title-]}

[# if options.compile_date == "none" -#]
  \date{}
[#- elif options.compile_date == "today" -#]
  \date{\today}
[#- elif options.compile_date == "frontmatter" -#]
  [#- if doc.date -#]
    \date{[-doc.date.day-] \DTMmonthname{[-doc.date.month-]}, [-doc.date.year-]}
  [#- else -#]
    \date{}  % MyST Frontmatter is missing a 'date' entry
  [#- endif -#]
[#- endif #]

[# for author in doc.authors #]
\author[
    [#-if author.corresponding-#],[#-endif-#]
    [--author.affiliations|join(', ', 'index')--]
    [#- if author.orcid -#] \orcid{[-author.orcid-]} [#- endif -#]
]{
    [--author.name-]
    [#- if author.corresponding -#]
    \thanks{
        Corresponding author
        [#-if author.email-#]
        : \texttt{[-author.email-]}
        [#- endif -#]
    }
    [#-endif-#]
}
[# endfor #]
[# for affiliation in doc.affiliations #]
%# As of MyST 1.6.0, one cannot specify institution & name: both map to 'name'. #%
%# For this reason, instead of (name, dept, inst), we do (dept, name)           #%
%# https://github.com/orgs/myst-templates/discussions/47                        #%
[#- set affil_els = [
                     (affiliation.value.department if affiliation.value.department),
                     (affiliation.value.name if affiliation.value.name),
                     ( affiliation.value.city if affiliation.value.city),
                     ( affiliation.value.state if affiliation.value.state),
                     ( affiliation.value.country if affiliation.value.country) 
                    ] -#]
\affil[[-affiliation.index-]]{[- affil_els | select() | join(", ") -]}
[# endfor #]

[# if parts.abstract #]
\abstract{[-parts.abstract-]}
[# endif #]

[# if doc.keywords #]
\keywords{[-doc.keywords|join(", ")-]}
[# endif #]

%%%%%%%%%%%%%%%  Main text   %%%%%%%%%%%%%%%

\begin{document}

\maketitle

[# if options.line_numbers #]
\linenumbers
[# endif #]

[-CONTENT-]

%%%%%%%%%%%%%%  Back matter  %%%%%%%%%%%%%%

[# if parts.data_availability #]
%%%%%%%%%%% Data availability %%%%%%%%%%%%
\section*{Data availability}
[- parts.data_availability -]
[# endif #]

[# if parts.code_availability #]
%%%%%%%%%%% Code availability %%%%%%%%%%%%
\section*{Code availability}
[- parts.code_availability -]
[# endif #]

[# if doc.bibliography #]
%%%%%%%%%%%%%%   Bibliography   %%%%%%%%%%%%%%
[# if options.use_biblatex #]
\emergencystretch 4pt   % Make it easier to typeset bibliographies without overflow
\printbibliography
\emergencystretch 2pt
[# else #]
\bibliography{[- doc.bibliography | join(", ") -]}
[# endif #]
[# endif #]

[# if parts.acknowledgments or parts.funding #]
%%%%%%%%%%%%%% Acknowledgements %%%%%%%%%%%%%%
\section*{Acknowledgements}
\footnotesize
[- parts.acknowledgments -]

[- parts.funding -]
\normalsize
[# endif #]

[# if parts.author_contributions #]
%%%%%%%%%%% Author contributions %%%%%%%%%%%%%
\section*{Author contributions}
\footnotesize
[- parts.author_contributions -]
\normalsize
[# endif #]

%%%%%%%%%%% Author contributions %%%%%%%%%%%%%
[# if options.competing_interests #]
\section*{Competing interests:} [-options.competing_interests-]
[# endif #]

%%%%%%%%%% External links and short-form information %%%%%%%%%%%
\section*{Additional information}
\footnotesize

[# if options.online_link #]
\paragraph{An online version} of this article is available at the following URL: \href{[-options.online_link-]}{[-options.online_link-]}
[# endif #]

[# if options.supplementary_link #]
\paragraph{Supplementary material} accompanies this paper at \href{[-options.supplementary_link-]}{[-options.supplementary_link-]}
[# endif #]

\footnotesize
\paragraph{Correspondence} should be addressed to
{[- doc.authors | selectattr("corresponding") | join(', ', 'name') -]}.
\normalsize

[# if parts.appendix #]
%%%%%%%%%%%%%%%%%%%%%%%%%%%% Appendix %%%%%%%%%%%%%%%%%%%%%%%%%%%%%%%%%
%# WORK IN PROGRESS! the \appendix + \supplementary combination      #%
%# is not yet extensively tested                                     #%
%# In most cases though only one is used                             #%

\appendix


[# if options.use_biblatex and options.separate_appendix_references -#]
\newrefsection
[# endif #]

[- parts.appendix -]

[# if options.use_biblatex and options.separate_appendix_references -#]
\emergencystretch 4pt   % Make it easier to typeset bibliographies without overflow
\printbibliography[title=Appended References]
\emergencystretch 2pt
[# endif #]

[# endif #]

[# if parts.supplementary #]
%%%%%%%%%%%%%%%%%%%%%%%%%%%% Supplementary %%%%%%%%%%%%%%%%%%%%%%%%%%%%%%%%%
%# In contrast to the Appendix, the Supplementary Information             #%
%# is intended to be submitted as a separate document.                    #%
%# We still compile everything as one document though:                    #%
%# - This is the easiest way to ensure all cross-references are correct.  #%
%# - It also ensures all the styling is consistent.                       #%
%# - For any other form of distribution                                   #%
%#   (preprint, personal communication, etc.) this is usually better.     #%
%# - It is easy to split the resulting PDF afterwards                     #%

\supplementary
\crefalias{figure}{sifigure}
\crefalias{table}{sitable}

[# if options.use_biblatex and options.separate_supplementary_references -#]
\newrefsection
[# endif #]

[- parts.supplementary -]

[# if options.use_biblatex and options.separate_supplementary_references -#]
\emergencystretch 4pt   % Make it easier to typeset bibliographies without overflow
\printbibliography[title=Supplementary References]
\emergencystretch 2pt
[# endif #]

[# endif #]

\end{document}
